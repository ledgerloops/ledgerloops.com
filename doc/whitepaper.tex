\documentclass[11pt,twoside,a4paper]{article}
\usepackage{listings}
\usepackage{color}

\definecolor{dkgreen}{rgb}{0,0.6,0}
\definecolor{gray}{rgb}{0.5,0.5,0.5}
\definecolor{mauve}{rgb}{0.58,0,0.82}

\lstset{frame=tb,
  language=Java,
  aboveskip=3mm,
  belowskip=3mm,
  showstringspaces=false,
  columns=flexible,
  basicstyle={\small\ttfamily},
  numbers=none,
  numberstyle=\tiny\color{gray},
  keywordstyle=\color{blue},
  commentstyle=\color{dkgreen},
  stringstyle=\color{mauve},
  breaklines=true,
  breakatwhitespace=true,
  tabsize=3
}

\begin{document}
\title{Ledger Loops: Debt loops across local ledgers (Work in Progress)}
\author{Michiel B. de Jong}
\date{November 2016}
\maketitle
\begin{abstract}
LedgerLoops is a protocol for clearing debts and credits across multiple ledgers. Unlike Bitcoin, which introduces one global public ledger, LedgerLoops acts on a loop of two or more local ledgers.
This whitepaper introduces the concept of cryptographically triggered debts, which is at the core of LedgerLoops, and explains how they can be represented in a LedgerLoops contract. It also defines a messaging protocol which can be used to send these contracts along the ledger loop in a peer-to-peer fashion, and a decentralized algorithm to find cycles in a debt graph.
\end{abstract}
\section{Cryptographically Triggered Debts}
A debt can be represented by an "I owe you asset A"-message ("IOU" for short), sent from the debtor to the creditor.
A cryptographically triggered debt would take the form "I owe you asset A from the moment person X signs a certain cleartext".
It doesn't matter who this person X is (they might be a complete stranger). In fact, person X is only identified by a public key. The cleartext holds no meaning either, and can be chosen at random; it only acts as a trigger to activate the debt.

A cryptographically triggered debt does not immediately give the receiver a claim to the asset that would be owed by the sender; however, if the receiver can present cryptographic proof that at any point in time,
person X signed the cleartext in question, this triggers the sender's debt into action.
So a cryptographically triggered debt, combined with a cryptographic proof that the cleartext it refers to has been signed by with the public key it refers to existed in the past, is equivalent to a first-order debt.
We say the existence of a valid signature {\em activates} all second-order promises that refer to this cleartext and this public key.

Settling a cryptographically triggerd debt (after it has been activated) works the same way as settling
a standard IOU: the IOU-receiver sends a message back to the IOU-sender, stating that the IOU-sender no longer owes
the assets mentioned in that specific debt statement.

\section{The LedgerLoops Contract Format (version 0.5)}
A LedgerLoops Contract is used to define cryptographically triggered debts. It refers to exactly one combination of a {\tt cleartext} and a {\tt pubkey}. It contains the following data (in UTF-8 JSON format):

\begin{lstlisting}
{
  "protocolVersion": "ledgerloops-0.5",
  "msgType": "cryptographically-triggered-contract",
  "keyAlgorithm": "ed25519", // TODO: add requirements about key format, hashing, padding, etc.
  "pubkey": <some public key, compatible with keyAlgorithm>,
  "cleartext": "Go!",
  "debtor": <a string which both parties understand to identify the debtor>,
  "beneficiary": <a string which both parties understand to identify the beneficiary>,
  "deliverable": <a string which both parties understand to define both the maximum and minimum conditions for considering this debt paid>
}
\end{lstlisting}

\section{The Algorithm}
To get a feel for how the LedgerLoops algorithm works, consider the following minimal example, involving three ledgers:

\begin{itemize}
\item Bob and Otto maintain a peer-to-peer ledger, on which Bob owes Otto a banana.
\item Otto and Michael maintain a peer-to-peer ledger, on which Otto owes Michael an orange.
\item Michael and Bob maintain a peer-to-peer ledger, on which Michael owes Bob a mango.
\end{itemize}

Bob, Otto, and Michael can each only see two of these three ledgers, so none of them know that a debt cycle
Bob -> Otto -> Michael -> back to Bob exists. To find this out and settle all three debts, they send the
following messages:

\subsection{Search round (debtor to creditor)}
Bob, Otto and Michael all send the following message to Otto, Michael and Bob, respectively:
"I have you as a creditor, but I also have at least one debtor, who may have other debtors, etc.,
which may eventually lead back to you."

In the actual implementation of the
search round, a second message type is used - one indicating "Yes, I'm interested in moving on to the probe round",
and one indicating "No, count me out". 
This round only establishes a best-effort statement of intent between peers, and its only
goal is to reduce the search space for the probe round.

\subsection{Probe round (debtor to creditor)}
Bob, Michael and Otto all send the following message to Michael, Otto and Bob, respectively:
"Right, so you said you have at least one debtor, try sending them this token, instructing them to forward it,
so I can see if it comes back to me." (accompanied by a long random string which is generated by Bob).

In the actual implementation, two tokens are used, one identifying Depth-First-Search tree, the other identifying
(backtrack-)paths along this tree.
After this round, if the token string used was long enough to be unguessable, and the token makes it back to Bob,
then Bob knows that a ledger loop exists (even though some participants may still pull out of the negotiations
in a later phase).

\subsection{Negotiation Round (debtor to creditor)}
Bob now generates an asymmetric cryptographic key pair, keeps the private key safe, and creates
a LedgerLoops contract mentioning:
\begin{itemize}
\item the public key from the keypair,
\item himself as the debtor,
\item Michael as the beneficiary,
\item as the deliverable he describes cancelling out Michael's mango debt.
\end{itemize}

Note that Bob does not intend to give Michael the equivalent of a mango, he instead states that he's willing to cancel
Michael's debt if this cryptographically triggered debt gets activated. So maybe 'cryptographically activated debt cancellation'
 is a better name than 'cryptographically triggered debt'.

Michael then sends an analoguous LedgerLoops contract to Otto (replacing the names and the asset referred to), and Otto does the
same again, so that Bob now has a CADC from Otto, and holds the private key that activates it.

This round is called the negotiation round because, in future versions, peers may get a chance to haggle over exchange rates. That
way, agents may for instance charge small transaction fees for converting between currencies - the smaller the transaction fee,
the higher the chance all peers in the ledger loop accept the offer. But for now, a transaction fee of zero works best. :)

\subsection{Settlement Round (creditor to debtor)}

In this round, the messages travel along the ledger loop in the opposite direction: from creditor to debtor. First, Bob signs the
cleartext, and presents Otto with the signature, thereby triggering Otto's contract and claiming his deliverable of cancelling out
Bob's banana debt.

Now that Otto has received the valid signature from Bob, he can use that to tell Michael to cancel out to Otto's orange debt.

Michael is also happy, because now that he received Bob's valid signature through Otto, he can tell Bob to cancel the mango debt.

All three debts have now been settled, and the goal has been achieved.

In this case the ledger length was quite short, but none of the participants know this precisely. Participants can roughly tell that
a loop where the rounds take a long time is more likely to have more participants than one on which message loop around quickly,
but this information is obfuscated by differences in speed of the communication network, and by participants who choose to wait a
little bit before forwarding a messages.

No central authority or Unique Node List was used, each participant only ever trusted their own direct peers, and none of the participants
exposed the contents of their private ledgers to the rest of the network.

Also, nobody except Bob knows who the public key belonged to that triggered all three LedgerLoops contracts, and each peer only exposed
their identity to their direct neighbors in the ledger loop, unless participants volunteer to make this public,
nobody (except for each participant's direct neighbors) knows exactly who particpated in which ledger loop.

\section{Conclusion}
This is a work in progress. Full implementation details of LedgerLoops 0.4, as well as an initial discussion of security considerations,
will be documented soon, once the code on
https://github.com/michielbdejong/ledgerloops stabilizes a bit.
\end{document}
